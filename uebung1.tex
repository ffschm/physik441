\documentclass[ngerman]{scrartcl}

\KOMAoptions{fontsize=11pt,paper=a4}
\KOMAoptions{DIV=14}

\usepackage{babel}
\usepackage[utf8]{inputenc}
\usepackage[T1]{fontenc}
\usepackage[autostyle=true]{csquotes}
\usepackage{amsmath}
\usepackage[varg]{txfonts}
\usepackage{graphicx}
\usepackage{hyperref}

\begin{document}

\title{Übungsblatt 1 zur Vorlesung 'Numerische Methoden der Physik' SS 2015}
\subtitle{Madelung-Energie des NaCl-Kristalls}
\author{Fabian Schmidt und Marvin Schmitz}
\maketitle

\newpage

\section*{Physikalische Beschreibung des Problems}

Wir betrachten ein NaCl-Kristallgitter. Wir wollen die Energie eines Ions im Kristallgitter numerisch bestimmen.
Dies wollen wir durch Summation der Einzelbeiträge erreichen: \\ \\
$ V_{i,j} = \frac{1}{4\pi \epsilon_0} \cdot \frac{e_ie_j}{r_i,j}\ \ mit\ \ r_{i,j} = a\cdot \sqrt{(i_1-j_1)^2+(i_2-j_2)^2+(i_3-j_3)^2} $\\ \\
Dabei ist a der Gitterabstand und $e_{i,j}$ jeweils positiv für ein Kation und negativ für ein Anion. Die gewünschte Genauigkeit der Approximation soll $10^{-5}$ sein. Desweiteren betrachten wir zusätzlich nochmal den Fall eines flachen Kristallgitters also $i_3=j_3$.

\section*{Modellierung des Problems}

Berechnet werden soll die Madelung-Energie. Dazu führen wir die Madelung-Konstante ein. Sei $E_{IP}$ die Energie eines Ions im Ionenpaar und $E_{IG}$ die Energie eines Ions im Gitter, dann ergibt sich:\\ \\
$ E_{IG} = n_1 \cdot \frac{E_{IP}}{c_1} + n_2\cdot \frac{E_{IP}}{c_2} + n_3\cdot \frac{E_{IP}}{c_3} + ... $ \\ \\
Wobei $c_1$ ein für den Abstand des Ions spezifische Konstante ist und $n_1$ die Häufigkeit des Ions (Quelle: Wikipedia Madelung-Konstante). Diese Konstanten kann man zusammenfassen zu der Madelung-Konstante:\\ \\
$ E_{IG} = \alpha \cdot E_{IP} $ 





\end{document}